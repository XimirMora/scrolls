
\documentclass{beamer}
\usetheme{metropolis}


\newcommand{\warn}[1]{\textcolor{red}{#1}}

\title{Metodi funzionali per valutare la conservazione evolutiva delle sequenze regolatrici}
% TODO check del titolo
% TODO logo unito
\date{}
\author{Relatore: Prof.\ Paolo Provero\\Candidato: Stefano Gilotto}
\institute{Universita' degli studi di Torino\\Dipartimento di Biotecnologie Molecolari e Scinze per la Salute}
\titlegraphic{\hfill\includegraphics[height=2cm]{logo-unito.png}}


\begin{document}
    \maketitle

    \begin{frame}[plain]{Introduzione}
        Il miglioramento delle tecniche di analisi ha posto nuovi alla luce una grande mole di dati.
        Questo ha posto la necessita' di prevedere quali sequenze siano attive, e quale sia la loro
        funzione.\\
        La conservazione evolutiva delle sequenze e' uno degli elementi su cui ci si basa maggiormente
        per applicare queste previsioni.
    \end{frame}


    \begin{frame}[plain]{Introduzione}
        Nonostante esistano numerose evidenze a favore dell'uso della conservazione,
        alcuni studi hanno individuato sequenze la cui funzionalita' e' conservata,
        ma non la loro sequenza.
        Sono presentate delle tecniche per poter prevedere queste sequenze attraverso
        analisi di carattere funzionale.
    \end{frame}

    \section{Fisher et al.}

    \begin{frame}[plain]{Fisher et al. (2006)}
        % TODO che informazioni mettere dell'articolo
        Ipotesi: la funzione regolatrice di regioni non-coding puo' essere conservata,
        in assenza di conservazione di sequenza.
        Oggetto di studio e' il \emph{locus} del gene \emph{Ret}, i cui esoni sono ben conservati,
        al contrario delle regioni limitrofe. Si confronta l'espressione delle regioni regolatrici
        tra i teleosti (Zebrafish) e i mammiferi (uomo).\\
    \end{frame}


    \begin{frame}[plain]{Fisher et al. (2006)}
        Le regione regolatrici sono definite attraverso il confronto di sequenze
        tra specie evolutivamente vicine.
        % TODO spiegare perche' si usano specie vicine?
        \begin{itemize}
            \item ZCS-Zebrafish Conserved Sequences:\\
            10 regioni conservate tra zebrafish e pesce palla
            \item HCS-Human Conserved Sequences:\\
            13 regioni conservate tra uomo e alcuni mammiferi
        \end{itemize}
    \end{frame}


    \begin{frame}[plain]{Fisher et al. (2006)}
        Le regioni regolatrici sono espresse in embrioni di zebrafish,
        attraverso costrutti transgenici con geni reporter.
        \begin{itemize}
            \item 9 su 10 ZCS hanno un espressione simile al gene \emph{Ret}
            \item 11 su 13 HCS hanno un espressione simile al gene \emph{Ret},\\
            anche in tessuti non presenti nei mammiferi o anatomicamente diversi.
        \end{itemize}
    \end{frame}


    \begin{frame}[plain]{Fisher et al. (2006)}
        \begin{figure}
            \includegraphics[scale=0.2]{RET1.jpg}
            \caption{Espressione delle ZCS e HCS}
        \end{figure}
        Le regioni regolatrici umane sono quindi funzionalmente analoghe a quelle di zebrafish.
    \end{frame}

    \section{Yao et al.}

    \begin{frame}[plain]{Yao et al. (2016)}
        Si possono individuare nuovi enhancers, basandosi su diverse caratteristiche funzionali
        legate alla loro espressione.\\
        In particolare si ricercano regioni che regolino lo sviluppo
        della nella zli-\emph{zona limitans intrathalamica},
        prendendo come punto di partenza il gene \emph{Ssh} e il relativo enhancer \textcolor{orange}{SBE1}.
    \end{frame}


    \begin{frame}[plain]{Yao et al. (2016)}
        Si cercano enhancers espressi in tessuti e momenti simili a SBE1.\\
        Vengono individuati 52 possibili enhancers e ne sono selezionati 7 di particolare rilevanza.\\
        Per comprendere quali siano le caratteristiche che li accomunino si
        cercano motifs e TFBS su questi.\\
    \end{frame}

    \begin{frame}[plain]{Yao et al. (2016)}
        Vengono individuati 6 motivi, che presentano sequenze capaci di potersi legare a TF noti.\\
        \begin{figure}
            \includegraphics[scale=0.45]{motifs.jpg}
        \end{figure}
    \end{frame}


    \begin{frame}[plain]{Yao et al. (2016)}
        I motifs sono soggetti ad analisi per confermare la loro funzione di TFBS,
        tutti con esito postivo
        \begin{itemize}
            \item saggio \emph{in vitro} con reporter luciferasi per enhancers con motifs deleti
            \item ChipSeq sui motifs e i TF
            \item saggio \emph{in vivo} con reporter LacZ per motifs mutati
        \end{itemize}
    \end{frame}


    \begin{frame}[plain]{Yao et al. (2016)}
        La delezione di SBE1 non elimna l'espressione di \emph{Ssh} nella zli,
        suggerendo la presenza di un altro enhancer.\\
        Basandosi sulle modificazioni istoniche arricchite in SBE1, viene individuato \textcolor{orange}{SBE5}.\\
        Tutte le verifiche sui TF applicate a SBE1 danno esito positivo per SBE5.\\
        La delezione di SBE1 e SBE5 \warn{silenzia} \emph{Ssh} nella zli.
    \end{frame}


    \begin{frame}[plain]{Yao et al. (2016)}
        Vengono usati i motifs individuati come base per una ricerca in \emph{S.kowalevskii}.\\
        \textcolor{orange}{skSBE1}: omologia di sequenza presente solo a livello dei 6 motifs
        Costrutti transgenici di skSBE1 in embrioni di topo hanno un pattern di espressione
        molto simile a mmSBE1. Lo stesso vale per mmSBE1 e mmSBE5 in \emph{S.kowalevskii}.
    \end{frame}


    \begin{frame}[plain]{Blow et al. (2010)}
        Il numero di enhancers attivi nello sviluppo del embrionale del cuore
        e' molto basso rispetto agli altri tessuti.\\
        ChipSeq con p300, coattivatore trascrizionale, espresso quasi
        ovunque nell'embrione di topo.
        \begin{itemize}
            \item 3597 regioni nel cuore
            \item 2759, 2786 e 3839 regioni nel prosencefalo,
            mesencefalo e negli arti
        \end{itemize}
    \end{frame}


    \begin{frame}[plain]{Blow et al.}
        Conservazione delle sequenze:
        \begin{itemize}
            \item 84\% delle regione predette nel tessuto cardiaco non sovrappongo quelle degli altri tessuti
            \item 6\% delle regioni predette nel tessuto cardiaco si sovrappongono a regioni `ultra-conserved'
            in PhastCons.\\
            In prosencefalo, mesencefalo e negli arti si ha il 44\%, 39\% e 30\% rispettivamente.
            \item le regioni del prosencefalo sono 7 volte piu' presenti tra quelle conservate
            tra mammiferi e pesci
        \end{itemize}
    \end{frame}


    \begin{frame}[plain]{Blow et al.}
        Test in vivo di 130 possibili enhancers predetti in cuore,
        usando un saggio transgenico in embrioni di topo:
        \begin{itemize}
            \item 81 su 130 sono enhancers attivi \underline{solo} nel cuore durante lo sviluppo
            \item arricchiti 13 volte nelle regioni di 10kb a monte dei geni definiti
            `heart-development' in Gene Ontology.
            \item arricchiti 14 volte delle regioni di 10kb a monte di 1000 geni espressi
            durante lo sviluppo embrionale del cuore.
        \end{itemize}
    \end{frame}


    \begin{frame}[plain]{Conclusione}
        La conservazione evolutiva delle sequenze rimane una grande risorsa
        per la predizione della funzionalita' delle sequenze.\\
        Ad oggi sono state osservate poche eccezzioni come quelle presentate,
        ma questo puo' essere dovuto al fatto che non le stiamo ricercando.\\
    \end{frame}

    \begin{frame}[plain]{Ringraziamenti}
        \begin{itemize}
            \item Prof.\ Paolo Provero
            \item Elena Grassi e il team dell'Unita Computazionale di Bioinformatica
            \item \warn{Gli altri}
        \end{itemize}
    \end{frame}


\end{document}
