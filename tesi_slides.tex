
\documentclass{beamer}
\usetheme{metropolis}
\usepackage[utf8x]{inputenc}

\newcommand{\warn}[1]{\textcolor{red}{#1}}
\newcommand{\hh}[1]{\textcolor{orange}{#1}}
\newcommand{\frag}[2]{\uncover<#1>{\item #2}}
% \long\def\frag[#1]#2{\uncover<#1>{\item #2}}

\title{Metodi funzionali per valutare la conservazione evolutiva delle sequenze regolatrici}
\date{}
\author{Relatore: Prof.\ Paolo Provero\\Candidato: Stefano Gilotto}
\institute{Università degli Studi di Torino\\Dipartimento di Biotecnologie Molecolari e Scienze per la Salute}
\titlegraphic{\hfill\includegraphics[height=2cm]{logo-unito.png}}


\begin{document}
    \maketitle

    % TODO add slide riguardo al problema del definire la

    \begin{frame}[plain]{Introduzione}
        Miglioramento delle tecniche di analisi:
        \begin{itemize}
            \frag{1-}{grande mole di dati di carattere biologico}
            \frag{2-}{funzionalità delle regioni non-coding e del `junk-DNA'}
            \frag{3-}{necessità di prevedere quali regioni siano funzionali}
        \end{itemize}
    \end{frame}

    \begin{frame}[plain]{Introduzione}
        modelli basati sulla conservazione delle sequenze
        \begin{itemize}
            \frag{2-}{non si basano su caratteristiche funzionali}
        \end{itemize}
        \begin{figure}
            \includegraphics[scale=0.6]{align.png}
        \end{figure}
    \end{frame}

    \begin{frame}[plain]{Introduzione}
        \begin{itemize}
            \frag{1-}{numerose evidenze a favore dell'utilizzo di modelli basati sulla conservazione}
            \frag{2-}{esistono sequenze la cui funzione è conservata, ma non la loro sequenza}
        \end{itemize}
        \uncover<3->{Sono presentate delle tecniche per poter prevedere e testare queste sequenze attraverso
        analisi di carattere funzionale.}
    \end{frame}


    \begin{frame}[plain]{Fisher et al.  }
        \includegraphics[scale=0.55]{ret-header.png}\\~\\
        \hh{Ipotesi}: la funzione regolatrice di regioni non-coding può essere conservata,
        in assenza di conservazione di sequenza.\\
        \hh{Oggetto di studio}: il \emph{locus} del gene \emph{Ret}, i cui esoni sono ben conservati,
        al contrario delle regioni limitrofe.\\
        \hh{Metodo}: Si confronta l'espressione delle regioni regolatrici
        tra i teleosti (Zebrafish) e i mammiferi (Uomo).\\
    \end{frame}


    \begin{frame}[plain]{Fisher et al.  }
        Le regione regolatrici sono definite attraverso il confronto di sequenze
        tra specie evolutivamente vicine.
        \begin{itemize}
            \item ZCS-Zebrafish Conserved Sequences:\\
            10 regioni conservate tra zebrafish e pesce palla
            \item HCS-Human Conserved Sequences:\\
            13 regioni conservate tra uomo e alcuni mammiferi
        \end{itemize}
        % TODO dire che queste regioni sono diverse tra di loro
    \end{frame}


    \begin{frame}[plain]{Fisher et al.  }
        Le regioni regolatrici sono espresse in embrioni di zebrafish,
        attraverso costrutti transgenici con geni reporter.
        \begin{itemize}
            \item 9 su 10 ZCS hanno un espressione simile al gene \emph{Ret}
            \item 11 su 13 HCS hanno un espressione simile al gene \emph{Ret},\\
            anche in tessuti non presenti nei mammiferi o anatomicamente diversi.
        \end{itemize}
    \end{frame}


    \begin{frame}[plain]{Fisher et al.  }
        \begin{figure}
            \includegraphics[scale=0.2]{RET1.jpg}
            % \caption{Espressione delle ZCS e HCS}
        \end{figure}
        Le regioni regolatrici umane sono quindi funzionalmente analoghe a quelle di zebrafish.
    \end{frame}

    % \section{Yao et al.}3

    \begin{frame}[plain]{Yao et al.  }
        \includegraphics[scale=0.6]{zli-header.png}\\~\\
        \hh{Ipotesi}: il controllo dello sviluppo della zli-\emph{zona limitans intrathalamica}
        precede la comparsa dei vertebrati.\\
        \hh{Oggetto di studio}: il sistema di controllo della zli e in particolare il gene \emph{Ssh}
        e il relativo enhancer \textcolor{cyan}{SBE1}.\\
        \hh{Metodo}: ricerca di nuovi enhancers attraverso caratteristiche funzionali
    \end{frame}


    \begin{frame}[plain]{Yao et al.  }
        Si cercano enhancers espressi in tessuti e fasi dello sviluppo embrionale simili a SBE1.\\
        \begin{itemize}
            \frag{1-}{vengono individuati 52 enhancers}
            \frag{2-}{si cercano motifs e TFBS su questi}
        \end{itemize}
    \end{frame}


    \begin{frame}[plain]{Yao et al.  }
        Vengono individuati 6 motifs, che presentano sequenze capaci di potersi legare a TF noti.\\
        \begin{figure}
            \includegraphics[scale=0.45]{motifs.jpg}
        \end{figure}
    \end{frame}


    \begin{frame}[plain]{Yao et al.  }
        I motifs sono soggetti ad analisi per confermare la loro funzione di TFBS,
        tutti con esito postivo
        \begin{itemize}
            \item saggio \emph{in vitro} con reporter luciferasi per enhancers con motifs deleti
            \item ChipSeq sui motifs e i TF
            \item saggio \emph{in vivo} con reporter LacZ per motifs mutati
        \end{itemize}
    \end{frame}


    \begin{frame}[plain]{Yao et al.  }
        La delezione di SBE1 non elimina l'espressione di \emph{Ssh} nella zli.
        \begin{itemize}
            \frag{1-}{utilizzando le modificazioni istoniche arricchite in SBE1
            è individuato \textcolor{cyan}{SBE5}}
            \frag{2-}{presenta gli stessi motifs di SBE1, ma non ci sono altre sequenze in comune}
            \frag{3-}{le analisi dei TFBS operate su SBE5 danno esito positivo}
        \end{itemize}
    \end{frame}


    \begin{frame}[plain]{Yao et al.  }
        Vengono usati i motifs individuati come base per una ricerca in \emph{S.kowalevskii}:\\
        \begin{itemize}
            \frag{1-}{\textcolor{magenta}{skSBE1}: omologia di sequenza presente solo a livello dei 6 motifs}
            \frag{2-}{\textcolor{magenta}{skSBE1} in embrioni di topo presenta
            il pattern di espressione di \textcolor{cyan}{mmSBE1}}
            \frag{3-}{\textcolor{cyan}{mmSBE1/5} in embrioni di \emph{S.kowalevskii}
             presentano il pattern di espressione di \textcolor{magenta}{skSBE1}}
        \end{itemize}
    \end{frame}

    % \section{Blow et al.  }

    \begin{frame}[plain]{Blow et al.  }
        \includegraphics[scale=0.6]{heart-header.png}\\~\\
        \hh{Ipotesi}: l'utilizzo della conservazione ha individuato pochi enhancers
        legati allo sviluppo cardiaco. Quindi questi potrebbero non essere conservati.\\
        \hh{Oggetto di studio}: sviluppo del cuore nell'embrione di topo.\\
        \hh{Metodo}: ChipSeq con p300, coattivatore trascrizionale.\\
        \begin{itemize}
            \item 3597 regioni nel cuore
            \item 2759, 2786 e 3839 regioni nel prosencefalo,
            mesencefalo e negli arti
        \end{itemize}
    \end{frame}


    \begin{frame}[plain]{Blow et al.  }
        Conservazione delle sequenze:
        \begin{itemize}
            \frag{1-}{84\% delle regione predette nel tessuto cardiaco non sovrappongo quelle degli altri tessuti}
            \frag{2-}{6\% delle regioni predette nel tessuto cardiaco si sovrappongono a regioni `ultra-conserved'
            in PhastCons.\\
            In prosencefalo, mesencefalo e negli arti si ha il 44\%, 39\% e 30\% rispettivamente.}
            \frag{3-}{le regioni del prosencefalo sono 7 volte più presenti tra quelle conservate
            tra mammiferi e pesci}
        \end{itemize}
    \end{frame}


    \begin{frame}[plain]{Blow et al.  }
        Test in vivo di 130 possibili enhancers predetti in cuore,
        usando un saggio transgenico in embrioni di topo:
        \begin{itemize}
            \frag{1-}{81 su 130 sono enhancers attivi \underline{solo} nel cuore durante lo sviluppo}
            \frag{2-}{arricchiti 13 volte nelle regioni di 10kb a monte dei geni definiti
            `heart-development' in Gene Ontology.}
            \frag{3-}{arricchiti 14 volte delle regioni di 10kb a monte di 1000 geni espressi
            durante lo sviluppo embrionale del cuore.}
        \end{itemize}
    \end{frame}

    % \section{Conclusione}

    \begin{frame}[plain]{Conclusione}
        La conservazione evolutiva delle sequenze rimane una grande risorsa
        per la predizione della funzionalità delle sequenze.
        \\~\\
        E' necessario ridefinire l'uso della conservazione
        delle sequenze per le regioni non-coding, basandosi maggiormente sulla funzionalità
        \\~\\
        Sottostimiamo il numero di regioni funzionali non conservate a livello di sequenza.
    \end{frame}


    \begin{frame}[plain]{Ringraziamenti}
        \begin{itemize}
            \item Prof.\ Paolo Provero
            \item Elena Grassi e il team dell'Unità Computazionale di Bioinformatica
        \end{itemize}
    \end{frame}


\end{document}
