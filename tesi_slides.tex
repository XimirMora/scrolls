
\documentclass{beamer}
\usetheme{metropolis}


\newcommand{\warn}[1]{\textcolor{blue}{#1}}
% \newcommand{\accent}[1]{\textcolor{orange}{#1}}

\title{Metodi funzionali per }
% TODO check del titolo
% TODO logo unito
% \date{\today} TODO
\author{Stefano Gilotto}
\institute{Universita' degli studi di Torino\\Dipartimento di Biotecnologie Molecolari e Scinze per la Salute}

\begin{document}
    \maketitle

    % NB mettere le sezioni?
    \begin{frame}{2REMOVE}
        \begin{itemize}
            \item introduzione alla necessita' di modelli predittivi,
                        con due parole sulla biologia
            \item limiti dei modelli predittivi?
            \item introduzione agli articoli
            \item Articoli
            \item Conclusione
            \item Ringraziamenti e affini
        \end{itemize}
    \end{frame}

    % ---------------------------
    \begin{frame}{Introduzione}
        * alcuni lavori hanno studiato regioni non conserved
    \end{frame}

    \begin{frame}{Introduzione}

    \end{frame}


    \begin{frame}{Fisher et al. (2006)}
        % TODO che informazioni mettere dell'articolo
        Viene analizzato il \emph{locus} del gene \emph{Ret}, i cui esoni
        sono ben conservati, al contrario delle regioni limitrofe.\\
        Si confronta l'espressione delle regioni regolatorie
        tra i teleosti (Zebrafish) e i mammiferi (uomo).\\
        % TODO appronfonodire la questione della tecnica usata
    \end{frame}


    \begin{frame}{Fisher et al. (2006)}
        Per definire le regioni regolatorie in maniera \emph{unbiased},
        si confrontano specie evolutivamente vicine.
        % TODO spiegare perche' si usano specie vicine?
        \begin{itemize}
            \item ZCS-Zebrafish Conserved Sequences:\\
                        10 regioni conservate tra zebrafish e pesce palla
            \item HCS-Human Conserved Sequences:\\
                        13 regioni conservate tra uomo e alcuni mammiferi
        \end{itemize}
    \end{frame}

    \begin{frame}{Fisher et al. (2006)}
        Le regioni regolatorie sono espresse in embrioni di zebrafish,
        attraverso costrutti transgenici con geni reporter.
        \begin{itemize}
            \item 9 su 10 ZCS hanno un espressione riconducibile al gene \emph{Ret}
            % NOTE cosa si intende per espressione riconducibile
            \item 11 su 13 HCS hanno un espressione riconducibile al gene \emph{Ret}
        \end{itemize}
    \end{frame}

    \begin{frame}{Fisher et al. (2006)}
        Le 11 HCS sono espresse anche in tessuti non presenti nei mammiferi (example)
        o fortemente diverse (example). \\
        Le regioni regolatorie umane sono quindi funzionalmente analoghe a quelle di zebrafish.
        % TODO insert image here
    \end{frame}

    % TODO discussione??

    \begin{frame}{Yao et al. (2016)}
        Si \warn{ricercano} nuovi enhancers, basandosi su diverse caratteristiche funzionali
        legate alla loro espressione, nella zli-\emph{zona limitans intrathalamica}.
        In particolare si \warn{ricercano} enhancers che regolino lo sviluppo di questa regione,
        prendendo come punto di partenza e' il gene \emph{Ssh} e il relativo enhancer \accent{SBE1}.
    \end{frame}

    \begin{frame}{Yao et al. (2016)}
        Si cercano enhancers espressi in tessuti e momenti simili a SBE1.\\
        Vengono individuati 52 possibili enhancers, di cui 7 particolarmente \warn{simili}.\\
        Per comprendere quali siano le caratteristiche che li accomunino si
        cercano motifs e TFBS su questi.\\
        Vengono individuati 6 motivi, che presentano sequenze capaci di potersi legare a TF noti.
        % TODO insert image
    \end{frame}

    \begin{frame}{Yao et al. (2016)}
        I motifs sono soggetti ad analisi per confermare la loro funzione di TFBS
        \begin{itemize}
            \item saggio \emph{in vitro} con reporter luciferasi per enhancers con motifs deleti
            \item ChipSeq sui motifs e i TF
            \item saggio \emph{in vivo} con reporter LacZ per motifs mutati
        \end{itemize}
    \end{frame}

    \begin{frame}{Yao et al. (2016)}
        La delezione di SBE1 non elimna l'espressione di \emph{Ssh} nella zli,
        suggerendo la presenza di un altro enhancer.\\
        Basandosi sulle modificazioni istoniche arricchite in SBE1, viene individuato \accent{SBE5}.\\
        Tutte le verifiche sui TF applicate a SBE1 danno esito positivo per SBE5.\\
        La delezione di SBE1 e SBE5 \warn{silenzia} \emph{Ssh} nella zli.
    \end{frame}

    \begin{frame}{Yao et al. (2016)}
        Vengono usati i motifs individuati come base per una ricerca in \emph{S.kowalevskii}.\\
        \accent{skSBE1}: omologia di sequenza presente solo a livello dei 6 motifs
        % TODO insert image here?
    \end{frame}

    \begin{frame}{Yao et al. (2016)}
        Costrutti transgenici di skSBE1 in embrioni di topo hanno un pattern di espressione
        molto simile a mmSBE1. Lo stesso vale per mmSBE1 e mmSBE5 in \emph{S.kowalevskii}.
        % TODO aggiungere la parte sulle altre specie? Non si e' mai parlato della questione della zli
    \end{frame}

    \begin{frame}{Blow et al. (2010)}
        Il numero di enhancers attivi nello sviluppo del embrionale del cuore
        e' molto basso rispetto agli altri tessuti.
        ChipSeq con p300, coattivatore trascrizionale, espresso quasi
        ovunque nell'embrione di topo.
        \begin{itemize}
            \item 3597 regioni nel cuore
            \item 2759, 2786 e 3839regioni nel prosencefalo,
            mesencefalo e negli arti
        \end{itemize}
    \end{frame}

    \begin{frame}{Blow et al.}
        Conservazione delle sequenze:
        \begin{itemize}
            \item 84\% delle regione predette \warn{in} cuore non sovrappongo quelle degli altri tessuti
            \item 6\% delle regioni predette in cuore si sovrappongono a regioni `ultra-conserved'
            in PhastCons.\\
            In proencefalo, mesencefalo e negli arti si ha il 44\%, 39\% e 30\% rispettivamente.
            \item le regioni del proencefalo sono 7 volte piu' presenti tra quelle conservate
            tra mammiferi e pesci
        \end{itemize}
    \end{frame}

    \begin{frame}{Blow et al.}
        Test in vivo di 130 possibili enhancers predetti in cuore,
        usando un saggio transgenico in embrioni di topo:
        \begin{itemize}
            \item 81 su 130 sono enhancers attivi \underline{solo} nel cuore durante lo sviluppo
            \item \warn{30 volte il numero predetto con la conservazione}
            \item arricchiti 13 volte nelle regioni di 10kb a monte dei geni definiti
            `heart-development' in Gene Ontology.
            \item arricchiti 14 volte delle regioni di 10kb a monte di 1000 geni espressi
            durante lo sviluppo embrionale del cuore.
        \end{itemize}
    \end{frame}


\end{document}
