
\documentclass{beamer}
\usetheme{metropolis}


\newcommand{\warn}[1]{\texcolor{blue}{#1}}

\title{Metodi funzionali per }
% TODO check del titolo
% TODO logo unito
% \date{\today} TODO
\author{Stefano Gilotto}
\institute{Universita' degli studi di Torino\\Dipartimento di Biotecnologie Molecolari e Scinze per la Salute}

\begin{document}
    \maketitle

    % NB mettere le sezioni?
    \begin{frame}{2REMOVE}
        \begin{itemize}
            \item introduzione alla necessita' di modelli predittivi,
                        con due parole sulla biologia
            \item limiti dei modelli predittivi?
            \item introduzione agli articoli
            \item Articoli
            \item Conclusione
            \item Ringraziamenti e affini
        \end{itemize}
    \end{frame}

    % ---------------------------
    \begin{frame}{Introduzione}
        * alcuni lavori hanno studiato regioni non conserved
    \end{frame}


    \begin{frame}{Fisher et al. (2006)}
        % TODO che informazioni mettere dell'articolo
        Viene analizzato il \emph{locus} del gene \emph{Ret}, i cui esoni
        sono ben conservati, al contrario delle regioni limitrofe.\\
        Si confronta l'espressione delle regioni regolatorie
        tra i teleosti(Zebrafish) e i mammiferi(uomo).\\
        % TODO appronfonodire la questione della tecnica usata
    \end{frame}


    \begin{frame}
        Per definire le regioni regolatorie in maniera \emph{unbiased},
        si confrontano specie evolutivamente vicine.
        % TODO spiegare perche' si usano specie vicine?
        \begin{itemize}
            \item ZCS-Zebrafish Conserved Sequences:\\
                        10 regioni conservate tra zebrafish e pesce palla
            \item HCS-Human Conserved Sequences:\\
                        13 regioni conservate tra uomo e alcuni mammiferi
        \end{itemize}
    \end{frame}

    \begin{frame}
        Le regioni regolatorie sono espresse in embrioni di zebrafish,
        attraverso costrutti transgenici con geni reporter.
        \begin{itemize}
            \item 9 su 10 ZCS hanno un espressione riconducibile al gene \emph{Ret}
            % NOTE cosa si intende per espressione riconducibile
            \item 11 su 13 HCS hanno un espressione riconducibile al gene \emph{Ret}
        \end{itemize}
    \end{frame}

    \begin{frame}
        Le 11 HCS sono espresse anche in tessuti non presenti nei mammiferi(example)
        o fortemente diverse(example). \\
        Le regioni regolatorie umane sono quindi funzionalmente analoghe a quelle di zebrafish.
        % TODO insert image here
    \end{frame}

    \begin{frame}{DISCUSSION??}

    \end{frame}

    \begin{frame}{Yao et al. (2016)}
        Si \warn{ricercano} nuovi enhancers, basandosi su diverse caratteristiche funzionali
        legate alla loro espressione, nella zli-\emph{zona limitans intrathalamica}.
        In particolare si \warn{ricercano} enhancers che regolino lo sviluppo di questa regione,
        prendendo come punto di partenza e' il gene \emph{Ssh} e il relativo enhancer \textcolor{orange}{SBE1}.
    \end{frame}

    \begin{frame}
        Si cercano enhancers espressi in tessuti e momenti simili a SBE1.\\
        Vengono individuati 52 possibili enhancers, di cui 7 particolarmente \warn{simili}.\\
        Per comprendere quali siano le caratteristiche che li accomunino si
        cercano motifs e TFBS su questi.\\
        Vengono individuati 6 motivi, che presentano sequenze capaci di potersi legare a TF noti.
    \end{frame}

    \begin{frame}
        l
    \end{frame}

\end{document}
